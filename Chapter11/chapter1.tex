\chapter{Equations}
\section{Quadratic equations}
\begin{dfn}
A quadratic equation is an equation of the form
\begin{equation}
  \label{quad}
  ax^2 + bx + c = 0
\end{equation}
where \( a, b \) and \( c \) are constants and \( a \neq 0 \).
\end{dfn}
\begin{thm}
A quadratic equation (\ref{quad}) has two solutions for the variable \( x \):
\begin{equation}
  \label{root}
  x_{1,2} = \frac{-b \pm \sqrt{b^2-4ac}}{2a}
\end{equation}
\end{thm}
\begin{dfn}
\label{disc}
The \emph{discrimimant} of a quadratic equation (\ref{quad})
is called \( \Delta \) and it is defined as:
\[
  \Delta = b^2 - 4ac
\]
\end{dfn}
\begin{lem}
If the discrimimant \( \Delta \) is zero, then the equation (\ref{quad}) has a double solution: (\ref{root}) becomes:
\[
  x = - \frac{b}{2a}
\]
\end{lem}
\begin{proof}
The equation \eqref{quad} together with the definition \ref{disc} becomes:
\begin{align*}
  x_{1,2} &= \frac{-b \pm \sqrt{b^2-4ac}}{2a} \\
          &= \frac{-b \pm \sqrt{\Delta}}{2a} \\
          &= \frac{-b \pm \sqrt{0}}{2a} \\
          &= -\frac{b}{2a} \\
\end{align*}
and there exists only one solution.
\end{proof}
